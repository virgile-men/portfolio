%-------------------------
% Resume in Latex
% Author : Jake Gutierrez
% Based off of: https://github.com/sb2nov/resume
% License : MIT
%------------------------

\documentclass[letterpaper,11pt]{article}

\usepackage{latexsym}
\usepackage[empty]{fullpage}
\usepackage{titlesec}
\usepackage{marvosym}
\usepackage[usenames,dvipsnames]{color}
\usepackage{verbatim}
\usepackage{enumitem}
\usepackage[hidelinks]{hyperref}
\usepackage{fancyhdr}
\usepackage[english]{babel}
\usepackage{tabularx}
\usepackage{fontawesome5}
\usepackage{multicol}
\setlength{\multicolsep}{-3.0pt}
\setlength{\columnsep}{-1pt}
\input{glyphtounicode}


%----------FONT OPTIONS----------
% sans-serif
% \usepackage[sfdefault]{FiraSans}
% \usepackage[sfdefault]{roboto}
% \usepackage[sfdefault]{noto-sans}
% \usepackage[default]{sourcesanspro}

% serif
% \usepackage{CormorantGaramond}
% \usepackage{charter}


\pagestyle{fancy}
\fancyhf{} % clear all header and footer fields
\fancyfoot{}
\renewcommand{\headrulewidth}{0pt}
\renewcommand{\footrulewidth}{0pt}

% Adjust margins
\addtolength{\oddsidemargin}{-0.6in}
\addtolength{\evensidemargin}{-0.5in}
\addtolength{\textwidth}{1.19in}
\addtolength{\topmargin}{-.7in}
\addtolength{\textheight}{1.4in}

\urlstyle{same}

\raggedbottom
\raggedright
\setlength{\tabcolsep}{0in}

% Sections formatting
\titleformat{\section}{
  \vspace{-4pt}\scshape\raggedright\large\bfseries
}{}{0em}{}[\color{black}\titlerule \vspace{-5pt}]

% Ensure that generate pdf is machine readable/ATS parsable
\pdfgentounicode=1

%-------------------------
% Custom commands
\newcommand{\resumeItem}[1]{
  \item\small{
    {#1 \vspace{-2pt}}
  }
}

\newcommand{\classesList}[4]{
    \item\small{
        {#1 #2 #3 #4 \vspace{-2pt}}
  }
}

\newcommand{\resumeSubheading}[4]{
  \vspace{-2pt}\item
    \begin{tabular*}{1.0\textwidth}[t]{l@{\extracolsep{\fill}}r}
      \textbf{#1} & \textbf{\small #2} \\
      \if\relax\detokenize{#3}\relax\else
        \textit{\small #3} & \textit{\small #4} \\
      \fi
    \end{tabular*}\vspace{-7pt}
}



\newcommand{\resumeSubSubheading}[2]{
    \item
    \begin{tabular*}{0.97\textwidth}{l@{\extracolsep{\fill}}r}
      \textit{\small#1} & \textit{\small #2} \\
    \end{tabular*}\vspace{-7pt}
}

\newcommand{\resumeProjectHeading}[2]{
    \item
    \begin{tabular*}{1.001\textwidth}{l@{\extracolsep{\fill}}r}
      \small#1 & \textbf{\small #2}\\
    \end{tabular*}\vspace{-7pt}
}

\newcommand{\resumeSubItem}[1]{\resumeItem{#1}\vspace{-4pt}}

\renewcommand\labelitemi{$\vcenter{\hbox{\tiny$\bullet$}}$}
\renewcommand\labelitemii{$\vcenter{\hbox{\tiny$\bullet$}}$}

\newcommand{\resumeSubHeadingListStart}{\begin{itemize}[leftmargin=0.0in, label={}]}
\newcommand{\resumeSubHeadingListEnd}{\end{itemize}}
\newcommand{\resumeItemListStart}{\begin{itemize}}
\newcommand{\resumeItemListEnd}{\end{itemize}\vspace{-5pt}}

%-------------------------------------------
%%%%%%  RESUME STARTS HERE  %%%%%%%%%%%%%%%%%%%%%%%%%%%%


\begin{document}

%----------HEADING----------
\begin{center}
    {\Huge \scshape Virgile Mendes} \\ \vspace{1pt}
    Montréal, QC \\ \vspace{1pt}
    \small %\raisebox{-0.1\height}\faPhone\ 514 478-2200 ~
    \href{mailto:contact@virgile.men}{\raisebox{-0.2\height}\faEnvelope\  \underline{contact@virgile.men}} ~ 
    \href{https://www.linkedin.com/in/virgile-men/}{\raisebox{-0.2\height}\faLinkedin\ \underline{linkedin.com/in/virgile-men}} ~
    \href{https://github.com/virgile-men}{\raisebox{-0.2\height}\faGithub\ \underline{github.com/virgile-men}} ~
    \href{https://virgile.men/}{\raisebox{-0.2\height}\faGlobe\ \underline{virgile.men}}
    \vspace{-8pt}
\end{center}

%-----------DIPLOMES-----------
\section{Diplômes}
  \resumeSubHeadingListStart
    \resumeSubheading
      {Baccalauréat en Génie Logiciel}{2025}
      {École de Technologie Supérieure (ÉTS)}{Montréal, QC}
    \resumeSubheading
    {Diplôme Universitaire de Technologie - Métiers du Multimédia et de l'Internet}{2021}
      {IUT Champs-sur-Marne, Université Gustave Eiffel}{France}
  \resumeSubHeadingListEnd

%-----------EXPERIENCE-----------
\section{Expériences}
  \resumeSubHeadingListStart

    \resumeSubheading
      {Auxiliaire d'enseignements}{Juillet 2025 -- Décembre 2025}
      {École de technologie supérieure}{Montréal, QC}
      \resumeItemListStart
        \resumeItem{Gérer 3 séances de laboratoire hebdomadaires pour les cours LOG210, LOG240 et GTI611 auprès de 200+ étudiants}
        \resumeItem{Accompagner 50+ équipes d'étudiants dans la réalisation de leurs travaux pratiques et projets}
        \resumeItem{Assurer la résolution de conflits Git et encadrer le travail en équipe pour améliorer la collaboration}
        \resumeItem{Évaluer 200+ rapports de laboratoire et 280+ soumissions de code par session en fournissant des retours constructifs}
        \resumeItem{\textit{\textbf{Mandat de refonte pour LOG240}} : migration vers une VM unique et adoption d'un GitLab auto-hébergé, substitution de Visual Paradigm par PlantUML, remplacement de TRAC par les outils intégrés à GitLab, intégration de pull requests avec modèles personnalisés, remplacement de UISpec4J par AssertJ Swing pour les tests end-to-end, mise à jour vers JUnit 5, ajout de githooks pour exécuter automatiquement les tests unitaires, et script simulant un pipeline CI/CD}
      \resumeItemListEnd

    \resumeSubheading
      {Développeur full-stack}{Janvier 2023 -- Juin 2025}
      {ISIOS}{À distance, QC}
      \resumeItemListStart
        \resumeItem{Corriger et résoudre plus de 50 anomalies mensuelles de la plateforme en TypeScript, PHP et SQL}
        \resumeItem{Développer et livrer 15+ fonctionnalités client selon les conditions d'acceptation et les normes de tests}
        \resumeItem{Améliorer les performances d'exportation de rapports (PDF, Excel) de 40\% en optimisant les requêtes SQL}
        \resumeItem{Refactoriser plus de 10 000 lignes de code back-end PHP conformément aux standards ISIOS}
        \resumeItem{Analyser les récits utilisateurs et implémenter 20+ tests end-to-end avec Playwright pour garantir la qualité}
        \resumeItem{\textit{\textbf{Réalisations R\&D de 2025}} : développement d'un agent AI à l'aide de LangGraph pour de l'extraction de données PDF propulsé par des LLM open-sources}
        \resumeItem{\textit{\textbf{Réalisations R\&D de 2023}} : automatisation complète des tests end-to-end via GitLab CI/CD, intégration de l'outil Matomo pour le suivi de l'expérience utilisateur, création d'une interface front-end connectée par API pour visualiser en temps réel des données issues de sources tierces, conception d'un module d'intelligence d'affaire (BI) en PHP basé sur le modèle ETL}
      \resumeItemListEnd

    \resumeSubheading
      {Analyste-programmeur}{Mai 2022 -- Décembre 2022}
      {Desjardins}{Montréal, QC}
      \resumeItemListStart
        \resumeItem{Développer un logiciel ETL en C\# .NET traitant 200+ enregistrements quotidiens et générant des rapports Excel comparant des données de 5+ sources externes}
        \resumeItem{Préparer la documentation accompagnant chaque programme et destinée aux usagers}
        \resumeItem{Faire de la programmation orientée objet et exécuter des stratégies de tests}
      \resumeItemListEnd

    
  \resumeSubHeadingListEnd
  
  

\vspace{-16pt}

%-----------TECHNICAL SKILLS-----------
\section{Compétences Techniques}
 \begin{itemize}[leftmargin=0.15in, label={}]
    \small{\item{
     \textbf{Langages de programmation}{: JavaScript (ES6+), TypeScript, PHP, Python, Java, C\#, Dart, SQL, HTML5, CSS3} \\
     \textbf{Frameworks et bibliothèques}{: React, Next.js, Node.js, FastAPI, Flutter, FlutterFlow, .NET, JUnit, Maven, Playwright, LangChain, LangGraph, LlamaIndex, smolagents, jQuery} \\
     \textbf{Outils DevOps et CI/CD}{: GitHub Actions, GitLab CI/CD, Docker, Docker Compose, Kubernetes, Dokploy, Uptime Kuma, Grafana} \\
     \textbf{Bases de données}{: MongoDB, PostgreSQL, MySQL, SQL Server, phpMyAdmin, ChromaDB} \\
     \textbf{Cloud et infrastructure}{: Cloudflare, OVH, Oracle Cloud, Raspberry Pi, Nextcloud, Linux, Windows, MacOS} \\
     \textbf{Design et UX}{: Adobe XD, Photoshop, Illustrator, Prototypage UI, Benchmark, Audit ergonomique, Tests utilisateurs} \\
     \textbf{IA et Data}{: LLM, RAG, OCR, ChromaDB, Ollama, Extraction d'exigences} \\
     \textbf{Outils et collaboration}{: GitHub, GitLab, Discord, Slack, Zoom, Google Sheets, Matomo} \\
    %  \textbf{Plateformes et CMS}{: WordPress, Prestashop, Elementor, Divi, WooCommerce} \\
     \textbf{Méthodologies et pratiques}{: Agile, Scrum, Kanban, TDD (Test-Driven Development), ETL, REST API, GraphQL, Git} \\
     \textbf{Compétences fonctionnelles}{: Architecture logicielle, Conception d'interfaces utilisateur, Intégration continue \& livraison continue, Documentation technique (normes APA), Outils collaboratifs (Confluence, Gantt), Référencement technique (SEO)} \\
    }}
 \end{itemize}
 \vspace{-16pt}

%-----------PROJECTS-----------
\section{Projets}
  \resumeSubHeadingListStart
    \resumeSubheading
      {Vue d'ensemble avec la sécurité des réseaux d'entreprise}{Mai 2025 -- Août 2025}
      {Projet d'études - GTI719, ÉTS}{Montréal, QC}
      \resumeItemListStart
        \resumeItem{Analyse de risque et processus d'audit selon Octave® Allegro}
        \resumeItem{Familiariser avec les fédérations d'identités (délégation d'authentification IdP, OAuth)}
        \resumeItem{Déploiement sécurisé de conteneurs Docker grâce à un orchestrateur Kubernetes}
      \resumeItemListEnd

    \resumeSubheading
      {Développement d'une application complexe de gestion de données}{Janvier 2025 -- Avril 2025}
      {Projet d'études - LOG660, ÉTS}{Montréal, QC}
      \resumeItemListStart
        \resumeItem{Analyse des besoins, conception du schéma relationnel, modélisation des règles d'affaires et insertion des données}
        \resumeItem{Développement d'une interface programme-BD à l'aide d'un framework de persistance transparente}
        \resumeItem{Optimisation de requêtes et analyse de la performance}
      \resumeItemListEnd

    \resumeSubheading
      {Conception et programmation d'un laboratoire de DevOps}{Septembre 2024 -- Décembre 2024}
      {Projet d'études - LOG680, ÉTS}{Montréal, QC}
      \resumeItemListStart
        \resumeItem{Automatisation de tests unitaire à travers des pre-commit}
        \resumeItem{Automatisation de tests d'intégration à travers un pipeline Github Action}
        \resumeItem{Déploiement continu d'une image Docker jusqu'à son déploiement dans un cluster Kubernetes}
        \resumeItem{Monitoring et visualisation de métriques à travers une instance Grafana}
      \resumeItemListEnd

    \resumeSubheading
      {Amélioration d'une architecture de microservices}{Septembre 2024 -- Décembre 2024}
      {Projet d'études - LOG430, ÉTS}{Montréal, QC}
      \resumeItemListStart
        \resumeItem{Intégration de microservices Docker pour comparer les trajets en autobus et voiture à l'aide de données GTFS et de l'API TomTom}
        \resumeItem{Mise en place de tactiques de redondance active et passive pour assurer la haute disponibilité des services}
        \resumeItem{Résilience du système aux défaillances via injection de fautes avec Chaos Monkey (pods computation, connecteurs, volumes Docker)}
        \resumeItem{Réplication des files de messages RabbitMQ avec quorum queues et reconnections automatiques avec ServiceMeshHelper}
        \resumeItem{Analyse de performance (throughput, stabilité, erreurs, temps de rétablissement)}
      \resumeItemListEnd

    \resumeSubheading
      {Hébergement de services}{Février 2024 -- Aujourd'hui}
      {Projet personnel - Raspberry Pi \& Cloudflare}{}
      \resumeItemListStart
        \resumeItem{Automatisation de workflow à travers n8n}
        \resumeItem{Migration de données Cloud sur Nextcloud}
        \resumeItem{Création de logiciels répondant à des problématiques avec des API de LLM (OpenAI, Gemini, etc)}
        \resumeItem{Gestion de l'ensemble des projets à travers de la containerisation Docker}
      \resumeItemListEnd

    \resumeSubheading
      {Hackathon Mobile Challenge}{Janvier 2024}
      {Projet personnel - ApplETS}{Montréal, QC}
      \resumeItemListStart
        \resumeItem{Conception d'une application mobile en 24h : \textit{Villes intelligentes pour un avenir durable}}
        \resumeItem{Prototypage d'UrbanEco, une application récompensant les comportements écoresponsables}
        \resumeItem{Apprentissage accéléré de Flutter dans un temps limité}
        \resumeItem{Gestion et travail collaboratif sous pression}
      \resumeItemListEnd
  \resumeSubHeadingListEnd

%-----------CERTIFICATIONS-----------
\section{Certifications}
  \resumeSubHeadingListStart
    \resumeSubheading
      {Candidat admissible à la Profession d'Ingénieur}{Depuis 2025}
      {Ordre des Ingénieurs du Québec}{QC, Canada}
    \resumeSubheading
      {AI Agents Course}{30 Juin 2025}
      {Hugging Face}{}
  \resumeSubHeadingListEnd

%-----------VOLUNTEERING-----------
\section{Bénévolat et comité}
  \resumeSubHeadingListStart
    \resumeSubheading
      {Comité institutionnel sur la santé mentale étudiante}{Septembre 2024 -- Juin 2025}
      {École de Technologie Supérieure}{Montréal, QC}
    \resumeSubheading
      {Accueillir, alphabétiser, Apprendre le français}{Septembre 2018 -- Mai 2019}
      {Lycée Saint Jean Baptiste de la Salle}{France}
  \resumeSubHeadingListEnd


\end{document}
